%%%%%%%%%%%%%%%%%%%%%%%%%%%%%%%%%%%%%%%%%
% Short Sectioned Assignment
% LaTeX Template
% Version 1.0 (5/5/12)
%
% This template has been downloaded from:
% http://www.LaTeXTemplates.com
%
% Original author:
% Frits Wenneker (http://www.howtotex.com)
% License:
% CC BY-NC-SA 3.0 (http://creativecommons.org/licenses/by-nc-sa/3.0/)
%
%%%%%%%%%%%%%%%%%%%%%%%%%%%%%%%%%%%%%%%%%

%----------------------------------------------------------------------------------------
%	PACKAGES AND OTHER DOCUMENT CONFIGURATIONS
%----------------------------------------------------------------------------------------

\documentclass[paper=a4, fontsize=11pt]{scrartcl} % A4 paper and 11pt font size

\usepackage[T1]{fontenc} % Use 8-bit encoding that has 256 glyphs
\usepackage{fourier} % Use the Adobe Utopia font for the document - comment this line to return to the LaTeX default
\usepackage[english]{babel} % English language/hyphenation
\usepackage{amsmath,amsfonts,amsthm} % Math packages
\usepackage{url}
\usepackage{sectsty} % Allows customizing section commands
\allsectionsfont{\centering \normalfont\scshape} % Make all sections centered, the default font and small caps

\usepackage{cite}
\usepackage{fancyhdr} % Custom headers and footers
\pagestyle{fancyplain} % Makes all pages in the document conform to the custom headers and footers
\fancyhead{} % No page header - if you want one, create it in the same way as the footers below
\fancyfoot[L]{} % Empty left footer
\fancyfoot[C]{} % Empty center footer
\fancyfoot[R]{\thepage} % Page numbering for right footer
\renewcommand{\headrulewidth}{0pt} % Remove header underlines
\renewcommand{\footrulewidth}{0pt} % Remove footer underlines
\setlength{\headheight}{13.6pt} % Customize the height of the header

\numberwithin{equation}{section} % Number equations within sections (i.e. 1.1, 1.2, 2.1, 2.2 instead of 1, 2, 3, 4)
\numberwithin{figure}{section} % Number figures within sections (i.e. 1.1, 1.2, 2.1, 2.2 instead of 1, 2, 3, 4)
\numberwithin{table}{section} % Number tables within sections (i.e. 1.1, 1.2, 2.1, 2.2 instead of 1, 2, 3, 4)

\setlength\parindent{11pt} % Removes all indentation from paragraphs - comment this line for an assignment with lots of text
\setlength{\parskip}{1em}
%----------------------------------------------------------------------------------------
%	TITLE SECTION
%----------------------------------------------------------------------------------------

\newcommand{\horrule}[1]{\rule{\linewidth}{#1}} % Create horizontal rule command with 1 argument of height

\title{	
\normalfont \normalsize 
\textsc{University of Derby, Department of Electronics, Mathematics \&\ Computing} \\ [25pt] % Your university, school and/or department name(s)
\horrule{0.5pt} \\[0.4cm] % Thin top horizontal rule
\huge The Impact of Reverberation Techniques on Immersion in Spatial Audio for Virtual Reality \\ % The assignment title
\horrule{2pt} \\[0.5cm] % Thick bottom horizontal rule
}

\author{Simon Durbridge} % Your name

\date{\normalsize\today} % Today's date or a custom date

\begin{document}

\maketitle % Print the title

%----------------------------------------------------------------------------------------
%	Introduction
%----------------------------------------------------------------------------------------

\section{Introduction}

Increases in available computing power, and great strides in research and development have brought a new surge of interest to virtual reality (VR) applications. 
With the introduction of improved VR systems, such as Google Jump, Oculus, Vive and facebook360, using both high end computers, and mobile phone technology to provide immersive visualisation; Further steps are being taken to provide users with immersive sound environments~\cite{OculusCo41online}. 
These environments may be created in an  attempt to emulate real places, or to characterise fictional places.\par

A significant part of how humans identify with their surroundings, includes the perception of the reverberant characteristics of the surrounding area~\cite{rumsey2012spatial}. 
Cues such as the timing and strength of early reflections, allow humans to perceive source size and direction, as well as how far from the nearest boundaries the perceivers are. 
Some early stage VR development platforms provide a very simplified model for how reverberation behaves in an audio system, and may be an oversimplification when attempting to create an immersive audio experience.\par

The aim of this report is to introduce a testing method, to allow for the evaluation of different reverberation methods with respect to immersive VR audio. 
Initially, some of the theory behind reverberation perception will be discussed. Following this, a brief discussion of current spatial audio techniques will be discussed. 
Finally, a testing framework will be proposed, in which subjects will evaluate different VR environments and reverb algorithms.

%------------------------------------------------
\section{Reverberation \&\ Perception}
%---------------------------------------------
\subsection{Reverberation}

Reverberation in a simplistic description, is the diffuse scattering of sound energy in a space due to the reflection of that energy from boundaries. Specifically, the level of these reflections are such as to balance in level with the ambient noise floor, at or beyond the critical distance from a source. That is in contrast with strong early or late reflections, however these strong reflections are of significant interest in this study. It is also worth noting that in this description, reverberation level is a function of source level, distance (and therefore time), and the absorption of sound energy in the problem geometry~\cite{Everest2009}.\\
A basic mathematical relationship between room characteristics, source level and reverberation time can be found below:\\
\begin{center}
$RT_{60} = \frac{0.161 A}{Sa}$
\end{center}



\begin{align}
A = 
\begin{bmatrix}
A_{11} & A_{21} \\
A_{21} & A_{22}
\end{bmatrix}
\end{align}

\subsection{Perception or Reverberation}
Aenean commodo ligula eget dolor. Aenean massa. Cum sociis natoque penatibus et magnis dis parturient montes, nascetur ridiculus mus. Donec quam felis, ultricies nec, pellentesque eu, pretium quis, sem.
\subsection{Reverberation Algorithms}

%------------------------------------------------
\section{Spatial Audio for Virtual Reality}
\subsubsection{Audio Spatial Perception}

Some Text.\\

\subsubsection{Ambisonics}
%\paragraph{Heading on level 4 (paragraph)}

Some Text.\\

%----------------------------------------------------------------------------------------
%	PROBLEM 2
%----------------------------------------------------------------------------------------

\section{Lists}

%------------------------------------------------

\subsection{Example of list (3*itemize)}
\begin{itemize}
	\item First item in a list 
		\begin{itemize}
		\item First item in a list 
			\begin{itemize}
			\item First item in a list 
			\item Second item in a list 
			\end{itemize}
		\item Second item in a list 
		\end{itemize}
	\item Second item in a list 
\end{itemize}

%------------------------------------------------

\subsection{Example of list (enumerate)}
\begin{enumerate}
\item First item in a list 
\item Second item in a list 
\item Third item in a list
\end{enumerate}

\begin{align} 
\begin{split}
(x+y)^3 	&= (x+y)^2(x+y)\\
&=(x^2+2xy+y^2)(x+y)\\
&=(x^3+2x^2y+xy^2) + (x^2y+2xy^2+y^3)\\
&=x^3+3x^2y+3xy^2+y^3
\end{split}					
\end{align}

%----------------------------------------------------------------------------------------

\section{References}

\bibliography{mylibrary}{}
\bibliographystyle{plain}

\end{document}